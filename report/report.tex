\documentclass[11pt, english, screen, research-development]{report-rd-info}
 
\usepackage[latin1]{inputenc}
   % - latin9, utf8, etc.
\usepackage[T1]{fontenc}


% some definitions for the actual contents
\usepackage{enumerate}
\usepackage{amsmath, amssymb}
\usepackage{algorithm}
\usepackage{algorithmic}

\usepackage{minted}
\usepackage{subfiles}

\usepackage{listings}
\usepackage{color}



\usepackage{hyperref}
\usepackage{cleveref}[2012/02/15]% v0.18.4; 
\crefformat{footnote}{#2\footnotemark[#1]#3}


\definecolor{dkgreen}{rgb}{0,0.6,0}
\definecolor{gray}{rgb}{0.5,0.5,0.5}
\definecolor{mauve}{rgb}{0.58,0,0.82}

\lstset{frame=tb,
  language=Java,
  aboveskip=3mm,
  belowskip=3mm,
  showstringspaces=false,
  columns=flexible,
  basicstyle={\small\ttfamily},
  numbers=none,
  numberstyle=\tiny\color{gray},
  keywordstyle=\color{blue},
  commentstyle=\color{dkgreen},
  stringstyle=\color{mauve},
  breaklines=true,
  breakatwhitespace=true,
  tabsize=3
}


% JavaScript
\lstdefinelanguage{JavaScript}{
  morekeywords={typeof, new, true, false, catch, function, return, null, catch, switch, var, if, in, while, do, else, case, break},
  morecomment=[s]{/*}{*/},
  morecomment=[l]//,
  morestring=[b]",
  morestring=[b]'
}

\begin{document}

\title{Integration of a JavaScript interpreter in DGtal}
%\subtitle{A Model and Short Guide}
\authorA{Lin}{SHEN}

\supervisor{Nicolas}{NORMAND}
 

\begin{abstract}
DGtal is a collaborative free project that aims to develop a set of efficient and generic algorithms in the field of discrete geometry. There is currently no tool to automate tasks or manipulate the library interactively. Thus, using DGtal requires the development of a ++ code, its compilation and then its execution. The goal of this project is to provide a JavaScript command interpreter for DGtal. After a long time of study and numerous experiments, we have achieved to create objects from DGtal and manipulate these objects by invoking the functions of DGtal in an environment of Node.js. Future work of this project is to automatically upgrade the JavaScript support to adapt to the evolution of DGtal.


\textit{Keywords}: DGtal, JavaScript, C++, interpreter, Node.js
    \begin{comment}
    
   As the name suggests, the abstract is a very short but informative piece of information about everything you did in this work, i.e., successively the description of the problem at hand, the objectives, the main point of the state-of-the-art, the choices made, the conceptual developments, the conducted experiments, results and interpretations, the new issues.

   It is the last thing to write!
   (Do not provide any details, give popular scientific information.  The introduction and conclusion of the report are there to develop the overall ideas.)
   \end{comment}
\end{abstract}

\begin{classification}
     \begin{comment}
%\small % Idem
   Bibliographic indexing is required. Use the ACM thesaurus:  See \url{http://www.acm.org/about/class/}. (The following example is based on the 1998 version, where general terms and additional key words are optional.)

   \category{H.2.8}{Database Applications}{Image databases}
   \category{H.3.3}{Information Search and Retrieval}{Clustering, Information filtering, Relevance feedback}
   \category{H.3.7}{Digital Libraries}{User issues}
   \category{I.5.3}{Clustering}{Algorithms, Similarity measures}
   \category{I.4.10}{Image Representation}{Statistical, Multidimensional}
   \terms{Algorithms, Performances, Experiments, Human factors, Verification.}
   \keywords{Content-based image retrieval system, Classification, Feedback loop, Supervised learning.}
   \end{comment}
\end{classification}
\begin{comment}
\begin{acknowledgements}
 
   The usual place of acknowledgements, if this pleases you or if the work has been conducted as part of a larger endeavour.
    
\end{acknowledgements}
\end{comment}
%--------------------------------------------------------------------------------




\maketitle

%--------------------------------------------------------------------------------


\newpage

\tableofcontents

%-----------------------------------------------------------
\chapter{Introduction}
\subfile{sections/introduction}


\chapter{Conceptions}
\subfile{sections/Conceptions}

\chapter{Proposals}
\subfile{sections/Proposals}

\chapter{Experiments and Results}
\subfile{sections/Experiments-and-Results}


\chapter{Conclusion}
\subfile{sections/Conclusion}

%-------------------------------------------------------------------------------

\bibliography{rapport}
%\listoffigures{}

%\listoftables{}

%\listofalgorithms{}

\appendix
%\chapter{Quoting Correctly}
%\subfile{appendix/Quoting-Correctly}

%\chapter{Reminder}
%\subfile{appendix/Reminder}

%\chapter{Detailed Measures}
%\subfile{appendix/Detailed-Measures}

%\chapter{Annotated References}
%\subfile{appendix/Annotated-References}


\chapter{Schedule}

\subfile{appendix/Schedule}


\chapter{Weekly Reports}
\subfile{appendix/Weekly-Reports}


\chapter{Self-assessment}
\subfile{appendix/Self-assessment}

\end{document}
