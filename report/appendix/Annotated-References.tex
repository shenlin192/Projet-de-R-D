\label{app:FichesLecture}

It is a good idea to provide a short summary of each of the papers in the references.

For each paper or book (in rare cases a \emph{complete and persistent} web site), you should extract the following kind of information:  firstly, a short summary;  then, your own analysis.

This is slightly redundant with the state-of-the-art chapter. However, notice that it lacks the overall synthesis, and it focuses on a single paper at a time.  In fact, this greatly helps to achieve the state-of-the-art.

\paragraph{\emph{Title of a paper}.}

Describe rapidly what problem is attacked in the paper (add the reference, e.g. \cite{Pascal-1671}).

\subparagraph{Summary.}

The summary presents the main idea of the paper and the work conducted \emph{by the authors} up to their conclusions

\subparagraph{Analysis.}

It is only during a second phase that you can analyse the contents of the paper, i.e., verify it (errors are always possible in the scientific literature!), express your opinion about the advances claimed by the paper, and establish a connection with your own work.

\paragraph{\emph{Title of a paper}.}

Etc.
