\label{chap:Proposals}

As mentioned in the chapter of introduction, we have already found 2 ways to connect a C++ library and NodeJs environment. One is to use the Node FFI model and the other to use SWIG. This chapter will firstly talk about the general idea of using Node FFI module and SWIG in our project. And then we will introduce Clang for analysing C++ files' abstract syntax tree, which may help to accomplish the automatically upgrading of JavaScript support when DGtal evolves. 

\section{Node FFI}

The general idea of using Node FFI to solve our problem is like:
\begin{itemize}
   \item Step 1: Create a C++ source file that contains or include all the functions that we may want to use lately.
   \item Step 2: Use the "make" tool to generate a shared object (a dynamic link library) for the source file that we created previously.
   \item Step 3: Invoke functions in the shared object from JavaScript files with the help of Node FFI model. \newline
\end{itemize}

By following the 3 steps above, we can successfully invoke C++ functions in JavaScript. It's very close to our objective. Create objects from DGtal would be done by invoking a constructor function. Manipulate the DGtal objects would be done by invoking algorithms of the library. The most difficult part is how to automate processes by executing complex commands or scripts to upgrade our JavaScript support when DGtal evolves. The Clang AST may be a solution. 


\section{SWIG}
The general idea of using SWIG is like:
    \begin{itemize}
       \item Step 1: Create a C++ source file just like step 1 of using Node FFI.
       \item Step 2: In order to use the functions defined or included by the source file in node, we simply need to write a SWIG interface file.
       \item Step 3: Create a binding file in order to generate the wrapped code.
       \item Step 4: Generate a node library by using command line
       \item Step 5: "require" the generated node library, and then we can invoke all the functions in the source file created by step 1 \newline
    \end{itemize}
    
The functionality of using SWIG is similar to Node FFI. However, the main advantage of using SWIG is the usage of a node library with JavaScript is much simpler than the usage of a FFI with JavaScript. Namely, the method has lower programming complexity. 


\section{Node FFI and SWIG Comparison}
Comparison of different aspects of the 2 possible solutions proposed above is given in table  \ref{tab:ProposalsComparison}

\begin{table}[H] % Use the starred version if the table is too large for a two-column document (screen version)
    \begin{center}
        \begin{tabular}{c|ccccc}
                                     & Node FFI & SWIG  \\
            \hline
            Programming Complexity   &          &$\surd$ \\  
            
            Basic warpping           &  $\surd$  &$\surd$ \\ 
            
            Support of Boost         & $\surd$  &$\approx$\\
            
            Support of template      &  $\surd$  &$\surd$ \\
                        
            
        \end{tabular}
    \end{center}
    \caption{Comparison of various proposals}
    \label{tab:ProposalsComparison}
\end{table}

\begin{comment}
If too many ideas have been enumerated and none can be totally and formally demonstrated to be the best, then some choice has probably to bemade in order to limit the time-consuming experiments that follow.  It is preferable to achieve a few well-conducted experiments that will provide unquestionable conclusions though in a limited scope.
\end{comment}


\section{Clang}
Clang is a powful compiler that can produced ASTs that closely resembles both the written C++ code and the C++ standard\footnote{\url{http://clang.llvm.org/docs/IntroductionToTheClangAST.html}}.

LibTooling is a library to support writing standalone tools based on Clang. Tools built with LibTooling, like Clang Plugins, can run FrontendActions over code\footnote{\url{http://clang.llvm.org/docs/LibTooling.html}}, which means we can find out the number of functions in a C++ file, signature of these functions, etc. So whenever the DGtal library evolves it's possible for us to detect its changes by analysing its AST and finally upgrade our JavaScript support files.


%--------------------------------------------------------------------------------

%\part{Experiments and Results} % in case of several chapters, then name the chapters differently
%\label{part:Experiments}