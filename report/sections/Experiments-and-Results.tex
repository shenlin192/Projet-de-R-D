
When it is not possible to work only at the theoretical level, experiments will take place.  In Computer Science, this is most often through developing code then running experiments with it and analysing the output.

Notice that if both parts are important, or if more than one proposal has been investigated, it is certainly better to split this chapter into several distinct chapters.  In the former case, you can even enclose these chapters inside a new part.

\section{Experiment 1}

Be extremely precise when describing your experiments.  They must be reproducible by any reader deeply interested in your work.

You should describe all the hypotheses and constraints related to the manipulated data and the developed algorithms or used softwares.  Check them carefully in order not to draw false conclusion from bad inputs or erroneous processing!

\begin{figure}
   \centering
   \setlength{\unitlength}{5cm}
   \begin{picture}(1.2, 1.3)
     \put(0, 0){\vector(1, 0){1.15}}
     \put(1.17, -.015){$x$}
     \put(0, 0){\vector(0, 1){1.15}}
     \put(0, 1.19){\makebox(0, 0){$y$}}
     % qbezier P1=(0.0/0.0) m1=0.0
     %         P2=(0.2998/0.905) m2=10.0
     \qbezier(0.0,0.0)(0.2093,0.0)(0.2998,0.905)
     % qbezier P1=(0.0/0.0) m1=0.0
     %         P2=(0.4625/0.8198) m2=5.0
     \qbezier(0.0,0.0)(0.2985,0.0)(0.4625,0.8198)
     % qbezier P1=(0.0/0.0) m1=0.0
     %         P2=(0.5757/0.744) m2=3.3333
     \qbezier(0.0,0.0)(0.3525,0.0)(0.5757,0.744)
     % qbezier P1=(0.0/0.0) m1=0.0
     %         P2=(0.6589/0.677) m2=2.5
     \qbezier(0.0,0.0)(0.3881,0.0)(0.6589,0.677)
     % qbezier P1=(0.0/0.0) m1=0.0
     %         P2=(0.7218/0.618) m2=2.0
     \qbezier(0.0,0.0)(0.4128,0.0)(0.7218,0.618)
     % qbezier P1=(0.0/0.0) m1=0.0
     %         P2=(0.7703/0.5662) m2=1.6667
     \qbezier(0.0,0.0)(0.4306,0.0)(0.7703,0.5662)
     % qbezier P1=(0.0/0.0) m1=0.0
     %         P2=(0.8081/0.5207) m2=1.4286
     \qbezier(0.0,0.0)(0.4436,0.0)(0.8081,0.5207)
     % qbezier P1=(0.0/0.0) m1=0.0
     %         P2=(0.8381/0.4806) m2=1.25
     \qbezier(0.0,0.0)(0.4536,0.0)(0.8381,0.4806)
     % qbezier P1=(0.0/0.0) m1=0.0
     %         P2=(0.862/0.4454) m2=1.1111
     \qbezier(0.0,0.0)(0.4611,0.0)(0.862,0.4454)
     % qbezier P1=(0.0/0.0) m1=0.0
     %         P2=(0.8814/0.4142) m2=1.0
     \qbezier(0.0,0.0)(0.4672,0.0)(0.8814,0.4142)
     % qbezier P1=(0.0/0.0) m1=0.0
     %         P2=(0.8972/0.3866) m2=0.9091
     \qbezier(0.0,0.0)(0.4719,0.0)(0.8972,0.3866)
     % qbezier P1=(0.0/0.0) m1=0.0
     %         P2=(0.9102/0.362) m2=0.8333
     \qbezier(0.0,0.0)(0.4758,0.0)(0.9102,0.362)
     \put(0.5757, 0.744){\circle*{.015}}
     \put(0.6,    0.74){$(u,v)$}
     \put(0.5,    0.05){$L$}
     \put(0.18,    0.45){$L$}
     \put(1, -.02){\line(0, 1){.04}}
     \put(1, .06){\makebox(0, 0){$L$}}
     \put(-.02, 1){\line(1, 0){.04}}
     \put(.03, .98){$L$}
   \end{picture}
   \caption[Quality of the results obtained]
           {Quality of the results obtained [Source : << \emph{Graphics in LaTeX 2$_\varepsilon$} >>, page~23]}
   \label{fig:Courbes}
\end{figure}

In this part of the report, you should put forth \emph{synthetic results}, in the form of several tables and figures (e.g., See figure~\ref{fig:Courbes}%
\footnote{Urs Oswald, \emph{``Graphics in LaTeX 2$_\varepsilon$,''} March 2003, \url{http://www.ursoswald.ch}.}%
)

For the sake of completeness, these synthetic results must be supported by the base results from which they are derived.  These lengthy results have to be put into the appendices (e.g., See Appendix~\ref{app:Measures}).

\section{Experiment 2}

Etc.

\section{Conclusion}

The conclusion of this chapter (alternatively of this part) is normally the culmination of this research work.  All the quality of the work, along with its limitations, have to be emphasised.  Also, you can draw some personal conclusion on the way this study has been conducted and what could be done better should you have the chance to restart it anew.

%--------------------------------------------------------------------------------